\documentclass{article}

\usepackage{fancyhdr}
\usepackage{extramarks}
\usepackage{amsfonts}
\usepackage{syntax}
\usepackage{stmaryrd}
\usepackage{mathpartir}
\usepackage{tipa}

%
% Basic Document Settings
%

\topmargin=-0.45in
\evensidemargin=0in
\oddsidemargin=0in
\textwidth=6.5in
\textheight=9.0in
\headsep=0.25in

\linespread{1.1}

\pagestyle{fancy}
\chead{Globally Ordered Type System}
\lfoot{\lastxmark}
\cfoot{\thepage}

\renewcommand\headrulewidth{0.4pt}
\renewcommand\footrulewidth{0.4pt}

\setlength\parindent{30pt}

\newcommand{\Z}{\mathbb{Z}}
\newcommand{\Zt}{$\Z$}

% Create a relational rule
% [#1] - Additional mathpartir arguments
% {#2} - Name of the rule
% {#3} - Premises for the rule
% {#4} - Conclusions for the rule
\newcommand{\relationRule}[4][]{\inferrule*[lab={\sc #2},#1]{#3}{#4}}

\newcommand{\rel}[1]{\ensuremath{\llbracket {#1} \rrbracket}}
\newcommand{\ttt}{\texttt}
\newcommand{\transform}{\rightsquigarrow}
\newcommand{\proj}{\pi}
\newcommand{\ttuple}{(\tau_1, \ldots, \tau_n)}
\newcommand{\etuple}{(e_1,\ldots,e_n)}
\newcommand{\bool}{\mathrm{bool}}
\newcommand{\integer}{\mathrm{int}}
\newcommand{\option}{\mathrm{option}}
\newcommand{\uoption}[1]{(\bool,#1)}
\newcommand{\opt}[1]{\texttt{opt-#1}}
\newcommand{\varOf}[1]{\texttt{varOf}(#1)}



\begin{document}
\section*{The System}
Everything in this document is defined with respect to an ordered set $\mathbb{G} = \{g_1, \dots, g_n\}$ of \emph{global variables}. Our goal is to enforce that $g_1$ is always used before $g_2$, $g_2$ before $g_3$, and so on. For simplicity, we require that each global variable be used; however, it is simple to relax this restriction by adding subtyping.

The \emph{effects} in this system specify which global variables have been used so far. Effects are represented as an integer from $1$ to $n+1$, where effect $i$ means that all global variables up to but not including $g_i$ have been used so far (that is, we expect to use $g_i$ next).

Unlike a regular ordered type system, we have a looser definition of when a variable is used. Our model is one in which the global variables are not considered \emph{used} until they are actually accessed. We can think of global variables as something like ref cells, and we wish to track their dereferences/assignments. This will be the syntax we use in this presentation; however, in actual dpt, the user views global variables as simple values. The dpt language will not have the dereference/assign operators that we will present below, but instead will have several builtin functions which perform similar tasks.

This system sits atop a regular type system for the language, and modifies it slightly. Specifically, we modify function types to include an effect. Instead of being represented as $\tau \rightarrow \tau$, they now have the form $(\tau, \epsilon) \rightarrow (\tau, \epsilon)$. We expect function arguments to be annotated with both, although in practice we intend to perform type inference.

We also allow users to define polymorphic (in effect) functions by annotating with an "effect" variable such as $\alpha$, analogously to regular polymorphic functions. As a result, we might wish to write a function with type $(\texttt{int}, \alpha) \rightarrow (\texttt{bool}, \alpha+1)$. Hence we shall bolster our effect grammar with effect variables and the + construct.

All in all, we have the following grammar of types and effects. For concreteness, we assume the base type Int and Bool, as well as function and pair types. We also need unit for ref cell updates.

$\forall \alpha. (\texttt{ref } (Int, \alpha), \alpha) \rightarrow (Int, \alpha+1)$

\begin{grammar}
	<$\epsilon$ (effects)> ::= 1 | 2 | \dots | n + 1 | $\alpha$ | $\epsilon + \epsilon$
	
	<$\tau$ (types)> ::= Unit | Int | Bool | ref ($\tau, \epsilon$) | $\tau * \tau$ | $\forall \alpha.(\tau, \epsilon) \rightarrow (\tau, \epsilon)$
\end{grammar}

Next, we define a simple functional language to use this system on:

\begin{grammar}
	<$x$ (variables)> ::= alphanumeric
	
	<$v$ (values)> ::= () | $\Z$ | True | False | $(v, v)$ | <closures> | $g_1$ | \dots | $g_n$
	
	<$e$ (expressions)> ::= $v$ | $x$ | $(e,e)$ | let $x$ = $e$ in $e$ | !$e$ | $e := e$ 
	\alt $\texttt{fun}\ [\alpha]\ (x:\tau, \epsilon) \rightarrow e$ | $e[\epsilon]$ $e$
\end{grammar} 

Note that we treat the global variables as values in their own right in this system.

The $\texttt{fun}\ [\alpha]$ syntax defines an effect-polymorphic function whose effects may use the universally quantified effect variable $\alpha$. The $e[\epsilon]$ $e$ syntax applies an effect-polymorphic function, instantiating $\alpha$ with $\epsilon$.

One might notice that updating a global variable is pointless, since it will thereafter count as used. In practice, we expect this system to be used on a series of programs, all of which share the global variables, and each of which uses the global variables in order. So a program might update a global variable for use in a later program.

\section*{The Typing Relation}

Type inference in this system simultaneously reasons about effects and regular types. As a result, the typing relation has an extra input and output, representing the state of the global variables before and after the expression, respectively.

Our typing relation will thus have the form $\Gamma, \epsilon \vdash e\ \colon \tau, \epsilon$. As usual, $\Gamma$ represents our current collection of bound variables; we treat it as a set. Our goal is that our program $p$ satisfies $\emptyset, 1 \vdash p\ \colon \tau, n+1$ for some $\tau$.

When writing rules, we use the convention that the metavariable $x$ matches all \emph{non-global} variables. We will always denote global variables using symbols of the form $g_i$.

Without further ado, let's start at the bottom. We write $\tau_i$ for the type of the value stored in $g_i$. We also assume an equivalence relation on effects such that addition is commutative and associative, and if $e_1, e_2$ are integers then the effect $e_1 + e_2$ is equivalent to the effect given by the integer sum of $e_1$ and $e_2$. We treat equivalent effects as equal for the purposes of applying the rules.

\begin{mathpar}
	\relationRule{int}{
		n \in \Z
	}{
		\Gamma, \epsilon \vdash n\ \colon \texttt{Int}, \epsilon
	}

	\relationRule{true}{
		\ 
	}{
		\Gamma, \epsilon \vdash \texttt{True}\ \colon \texttt{Bool}, \epsilon
	}

	\relationRule{false}{
		\ 
	}{
		\Gamma, \epsilon \vdash \texttt{False}\ \colon \texttt{Bool}, \epsilon
	}

	\relationRule{local variable}{
		\Gamma[x] = \tau
	}{
		\Gamma, \epsilon \vdash x\ \colon \tau, \epsilon
	}

	\relationRule{global variable}{
		\
	}{
		\Gamma, \epsilon \vdash g_i\ \colon \texttt{ref}(\tau_i, i), \epsilon
	}
\end{mathpar}

Notice how none of these rules increment $\epsilon$. We can now write some of the simpler recursive rules:

\begin{mathpar}
	\relationRule{pair}{
		\Gamma, \epsilon \vdash e_1\ \colon \tau_1, \epsilon_1\\
		\Gamma, \epsilon_1 \vdash e_2\ \colon \tau_2, \epsilon_2
	}{
		\Gamma, \epsilon \vdash (e_1, e_2)\ \colon \tau_1 * \tau_2, \epsilon_2
	}

	\relationRule{let}{
		\Gamma, \epsilon \vdash e_1\ \colon \tau_1, \epsilon_1\\
		\Gamma[x := \tau_1], \epsilon_1 \vdash e_2\ \colon \tau_2, \epsilon_2
	}{
		\Gamma, \epsilon \vdash \texttt{let } x = e_1 \texttt{ in } e_2\ \colon \tau_2, \epsilon_2
	}
\end{mathpar}

Now let's finally write some rules that actually change $\epsilon$

\begin{mathpar}
	\relationRule{deref}{
		\Gamma, \epsilon \vdash e\ \colon \texttt{ref}(\tau, \epsilon_1), \epsilon_1
	}{
		\Gamma, \epsilon \vdash !e\ \colon \tau, \epsilon_1+1
	}

\relationRule{ref-assign}{
	\Gamma, \epsilon \vdash e_1\ \colon \tau, \epsilon_1\\
	\Gamma, \epsilon_1 \vdash e_2\ \colon \texttt{ref}(\tau, \epsilon_2), \epsilon_2
}{
	\Gamma, \epsilon \vdash e_2 := e_1\ \colon \texttt{Unit}, \epsilon_2+1
}
\end{mathpar}

Whoof. Notice how, when we evaluate the ref expression, its carried $\epsilon$ is expected to be equivalent to the output of the typing relation. Since we're about to use it, this ensures we're actually ready and able to access it. Also note how $e_2$ and $e_1$ appear "backwards" in the ref-assign rule, to reflect the order in which they are evaluated.

Now, finally, let's have some \texttt{fun}.
\begin{mathpar}
	\relationRule{abs}{
		\Gamma[x := \tau], \epsilon_f \vdash e\ \colon \tau_1, \epsilon_1
	}{
		\Gamma, \epsilon \vdash \texttt{poly } \alpha (x : \tau, \epsilon_f) \rightarrow e\ \colon \forall \alpha.(\tau, \epsilon_f) \rightarrow (\tau_1, \epsilon_1), \epsilon
	}
	
	\relationRule{app}{
		\Gamma, \epsilon \vdash e_1\ \colon \forall \alpha.\tau_1, \epsilon_1\\
		\Gamma, \epsilon_1 \vdash e_2\ \colon \tau_2, \epsilon_2\\
		\tau_1[\epsilon_\alpha/\alpha] = (\tau_2, \epsilon_2) \rightarrow (\tau', \epsilon')
	}{
		\Gamma, \epsilon \vdash e_1[\epsilon_\alpha]\ e_2\ \colon \tau', \epsilon'
	}
\end{mathpar}

Where we assume the substitution relation on effects works as expected.

\section*{Weak Globally Ordered Type System}
It's pretty easy to see how we could convert this to a weakly-ordered effect system (in which each ordered variable may be used at most once, instead of exactly once) just by inserting some inequalities into the rules. More interesting, though, is the observation that this corresponds to adding subtyping to the effects. The effect rules are:

\begin{mathpar}
	\relationRule{refl-effect}{
		\ 
	}{
		\epsilon <: \epsilon
	}

	\relationRule{trans-effect}{
		\epsilon_1 <: \epsilon_2\\
		\epsilon_2 <: \epsilon_3
	}{
		\epsilon_1 <: \epsilon_3
	}

	\relationRule{sub-effect}{
		i < j
	}{
      	i <: j
	}

	\relationRule{plus-effect}{
		i <: j\\ k <: l
	}{
		i+k <: j+l
	}
\end{mathpar}

While the type rules are:

\begin{mathpar}
	\relationRule{refl-type}{
		\ 
	}{
		\tau <: \tau
	}
	
	\relationRule{trans-type}{
		\tau_1 <: \tau\
		\tau_2 <: \tau_3
	}{
		\tau_1 <: \tau_3
	}

	\relationRule{function}{
		\tau_1' <: \tau_1\\
		\tau_2 <: \tau_2'\\
		\epsilon_1' <: \epsilon_1\\
		\epsilon_2 <: \epsilon_2'
	}{
		(\tau_1, \epsilon_1) \rightarrow (\tau_2, \epsilon_2) <: (\tau_1', \epsilon_1') \rightarrow (\tau_2', \epsilon_2')
	}

	\relationRule{ref}{
		\epsilon_1 <: \epsilon_2
	}{
		ref (\tau, \epsilon_1) <: ref (\tau, \epsilon_2)
	}
\end{mathpar}

Finally, we augment our typing relation with the single rule
\begin{mathpar}
	\relationRule{subtyping}{
		\tau_1 <: \tau_2\\
		\epsilon_1 <: \epsilon_2\\
		\Gamma, \epsilon \vdash e\ \colon\ \tau_1, \epsilon_1
	}{
		\Gamma, \epsilon \vdash e\ \colon\ \tau_2, \epsilon_2
	}
\end{mathpar}

and all other standard subtyping rules (refl, trans, function subtyping.)

\section*{Operational Semantics}





\end{document}
%%% Local Variables:
%%% mode: latex
%%% TeX-master: t
%%% End: